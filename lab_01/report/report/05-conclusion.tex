\phantomsection\section*{ЗАКЛЮЧЕНИЕ}\addcontentsline{toc}{section}{ЗАКЛЮЧЕНИЕ}

Цель данной работы была достигнута, а именно, были изучены, реализованы и исследованы алгоритмы поиска расстояний Левенштейна и Дамерау~---~Левенштейна.

Для достижения поставленной цели были решены следующие задачи:
\begin{enumerate}
    \item описаны алгоритмы поиска расстояний Левенштейна и Дамерау~---~Левенштейна;
    \item обоснован выбор средств реализации алгоритмов;
    \item реализован итерационный алгоритм нахождения расстояния Левенштейна,
    \item реализован итерационный алгоритм нахождения расстояния Дамерау --- Левенштейна,
    \item реализован рекурсивный алгоритм нахождения расстояния Дамерау --- Левенштейна,
    \item реализован рекурсивный с кешированием алгоритм нахождения расстояния Дамерау --- Левенштейна;
    \item проведён сравнительный анализ алгоритмов по критериям используемого процессорного времени и максимальной затрачиваемой памяти;
    \item описаны и проанализированы полученные результаты в отчёте.
\end{enumerate}

В результате исследования выяснилось, что наиболее эффективной по времени выполнения является итерационная реализация алгоритма нахождения расстояния Левенштейна.
Она превосходит итерационную реализацию алгоритма нахождения расстояния Дамерау~---~Левенштейна в $1.1$ раз, рекурсивную с кешированием --- в $1.5$ раз, рекурсивную без кеширования --- в $865442$ раза на строках длиной $12$ символов.

Наименее затратным по памяти оказалась рекурсивная без кеширования реализация алгоритма нахождения расстояния Дамерау~---~Левенштейна.
Она расходует память пропорционально сумме длин входящих строк, в то время как остальные реализации расходуют память пропорцонально произведению длин входящих строк.
