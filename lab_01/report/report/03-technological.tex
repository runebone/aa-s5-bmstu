\section{Технологическая часть}

В данном разделе будут описаны требования к программному обеспечению, средства реализации и функциональные тесты.
Также будут приложены листинги кода.

\subsection{Средства реализации}

Для реализации данной работы был выбран язык \texttt{C++} \cite{isocpp} по следующим причинам:
\begin{itemize}
    \item имеется опыт разработки на данном языке;
    \item в стандартной библиотеке языка присутствует функция \texttt{clock\_gettime} \cite{cpptime}, которая позволяет рассчитать процессорное время конкретного потока;
    \item наличие библиотеки для проведения автоматизированного тестирования \texttt{gtest} \cite{gtest};
    \item язык поддерживается отладчиком \texttt{gdb} \cite{gdb}.
\end{itemize}

Для построения графиков был выбран язык \texttt{Python} \cite{python}, так как в нём есть все необходимые для этого библиотеки \cite{pd} \cite{np} \cite{mpl}.

В качестве среды разработки был выбран \texttt{Neovim} \cite{nvim}. %$\vcenter{\hbox{\includegraphics[scale=0.03]{img/nvim.png}}}$.

\subsection{Сведения о модулях программы}

Программа поделена на следующие модули:
\begin{itemize}
    \item \texttt{main.cpp} --- файл, содержащий точку входа в программу, из которой происходит вызов алгоритмов по разработанному интерфейсу;
    \item \texttt{func.cpp} --- файл, содержащий реализации алгоритмов нахождения расстояний Левенштейна и Дамерау --- Левенштейна;
    \item \texttt{test.cpp} --- файл, содержащий модульные тесты и точку входа в программу для их запуска;
    \item \texttt{benchmark.cpp} --- файл, содержащий функции для проведения замеров времени работы реализаций алгоритмов;
    \item \texttt{globals.cpp} --- файл, содержащий глобальные переменные для замеров количества рекурсивных вызовов и максимального размера стека;
    \item \texttt{plot.py} --- файл, содержащий программу для построения графиков по данным о времени выполнения реализаций алгоритмов.
\end{itemize}

\subsection{Реализация алгоритмов}

Листинги исходных кодов программ \ref{lst:cm} -- \ref{lst:dlrec} приведены в приложении.

\subsection{Функциональные тесты}

В таблице \ref{tab:ft} приведены функциональные тесты для алгоритмов нахождения расстояний Левенштейна и Дамерау --- Левенштейна. Все тесты пройдены успешно.

\begin{table}[H]
	\small
	\begin{center}
		\begin{threeparttable}
		\caption{Функциональные тесты}
		\label{tab:ft}
		\begin{tabular}{|c|c|c|c|c|c|}
			\hline
			\multicolumn{2}{|c|}{\bfseries Входные данные}
			& \multicolumn{4}{c|}{\bfseries Расстояние и алгоритм нахождения расстояния} \\ 
			\hline 
			&
			& \multicolumn{1}{c|}{\bfseries Левенштейна} 
			& \multicolumn{3}{c|}{\bfseries Дамерау~---~Левенштейна} \\ \cline{3-6}
			
			\bfseries Строка 1 & \bfseries Строка 2 & \bfseries Итерационный & \bfseries Итерационный
			
			& \multicolumn{2}{c|}{\bfseries Рекурсивный} \\ \cline{5-6}
			& & & & \bfseries Без кеша & \bfseries С кешем \\
			\hline
			$\lambda$ & $\lambda$ & 0 & 0 & 0 & 0 \\
			\hline
			$abc$ & $\lambda$ & 3 & 3 & 3 & 3 \\
			\hline
			$\lambda$ & $a$ & 1 & 1 & 1 & 1 \\
			\hline
			$a$ & $a$ & 0 & 0 & 0 & 0 \\
			\hline
			$a$ & $b$ & 1 & 1 & 1 & 1 \\
			\hline
			$\xi\eta$ & $\eta\xi$ & 2 & 1 & 1 & 1 \\
			\hline
			$abcabcab$ & $bacbacba$ & 6 & 3 & 3 & 3 \\
			\hline
			$python$ & $ptyhon$ & 2 & 1 & 1 & 1 \\
			\hline
			скат & кот & 3 & 2 & 2 & 2 \\
			\hline
		\end{tabular}	
		\end{threeparttable}
	\end{center}
\end{table}

\newpage

\subsection*{Вывод}

Были реализованы и протестированы алгоритмы:
\begin{itemize}
    \item Итерационный алгоритм нахождения расстояния Левенштейна;
    \item Итерационный алгоритм нахождения расстояния Дамерау --- Левенштейна;
    \item Рекурсивный алгоритм нахождения расстояния Дамерау --- Левенштейна;
    \item Рекурсивный с кешированием алгоритм нахождения расстояния Дамерау --- Левенштейна.
\end{itemize}
