\section{Аналитическая часть}

\subsection{Битонная сортировка}

% Своими словами описать, что делает каждый алгоритм

\subsection{Пузырьковая сортиорвка}

Пузырьковая сортировка происходит засчёт многократного прохождения по элементам массива и последовательного сравнения соседних элементов.
На каждой итерации соседние элементы меняются местами, если они они не соответствуют нужному порядку.
Если была выбрана сортировка по возрастанию, легко проверить, что в результате первого прохождения по всем элементам массива, максимальный элемент будет гарантированно помещён в его конец.
На второй итерации второй по величине элемент будет помещён на предпоследнее место и т.д.
То есть, при массиве размера $N$ на итерации $i$ будут отсортированы последние $i$ элементов, значит, можно не проходить каждый раз весь массив целиком, а лишь его первые $N - i$ элементов.
Также, если в результате прохождения по массиву не было произведено никаких перестановок, массив уже является отсортированным.

\subsection{Поразрядная сортировка}

Существует несколько видов поразрядной сортировки --- побитовая и целочисленная.
В данной работе будет рассмотрена побитовая поразрядная сортировка.

Побитовая поразрядная сортировка происходит засчёт прохождения по всем элементам массива побитово, от младшего бита --- к старшему.
На каждой итерации элементы исходного массива распределяются между двумя массивами --- массив элементов, у которых текущий бит равен нулю, и массив элементов, у которых текущий бит равен единице.
После распределения элементов между двумя массивами, массивы объединяются.
Если выполнялась сортировка по возрастанию, первыми в объединённом массиве будут находится элементы, у которых текущий бит был равен нулю.
После объединения массивов счётчик текущего бита увеличивается, и цикл повторяется.
После обработки последнего бита элементов массива, массив оказывается отсортированным.

\subsection*{Вывод}

% Были рассмотрены алгоритмы, немного воды
