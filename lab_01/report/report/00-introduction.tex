\phantomsection\section*{ВВЕДЕНИЕ}\addcontentsline{toc}{section}{ВВЕДЕНИЕ}

Редакционные расстояния --- расстояние Левенштейна и его модификация --- расстояние Дамерау --- Левенштейна --- метрики cходства между двумя символьными последовательностями.
Расстоянием в этих метриках считается минимальное число односимвольных преобразований (удаления, вставки, замены или транспозиции), необходимых для преобразования одной последовательности символов в другую.

Редакционные расстояния применяются, например, в следующих ситуациях:
\begin{itemize}
    \item для исправления ошибок в слове поискового запроса;
    \item в формах заполнения информации на сайтах;
    \item для распознавания рукописных символов;
    \item в базах данных \cite{ldla}.
\end{itemize}

Целью данной лабораторной работы является реализация и исследование алгоритмов поиска расстояний Левенштейна и Дамерау --- Левенштейна.

Для достижения поставленной цели нужно решить следующие задачи:
\begin{enumerate}
    \item описать алгоритмы поиска расстояний Левенштейна и Дамерау~---~Левенштейна;
    \item обосновать выбор средств реализации алгоритмов;
    \item реализовать итерационный алгоритм нахождения расстояния Левенштейна;
    \item реализовать итерационный алгоритм нахождения расстояния Дамерау~---~Левенштейна;
    \item реализовать рекурсивный алгоритм нахождения расстояния Дамерау~---~Левенштейна;
    \item реализовать рекурсивный с кешированием алгоритм нахождения расстояния Дамерау~---~Левенштейна;
    \item провести сравнительный анализ реализаций алгоритмов по используемому процессорному времени и максимальной затрачиваемой памяти;
    \item описать и проанализировать полученные результаты в отчёте.
\end{enumerate} % TODO tests, block schemes formal language etc.
