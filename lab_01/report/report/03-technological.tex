\section{Технологическая часть}

В данном разделе будут описаны требования к программному обеспечению, средства реализации и функциональные тесты.
Также будут приложены листинги кода.

\subsection{Средства реализации}

Для реализации данной работы был выбран язык \texttt{C++} \cite{isocpp} по следующим причинам:
\begin{itemize}
    \item имеется опыт разработки на данном языке;
    \item в стандартной библиотеке языка присутствует функция \texttt{clock\_gettime} \cite{cpptime}, которая позволяет рассчитать процессорное время конкретного потока;
    \item наличие библиотеки для проведения автоматизированного тестирования \texttt{gtest} \cite{gtest};
    \item язык поддерживается отладчиком \texttt{gdb} \cite{gdb}.
\end{itemize}

Для построения графиков был выбран язык \texttt{Python} \cite{python}, так как в нём есть все необходимые для этого библиотеки \cite{pd} \cite{np} \cite{mpl}.

В качестве среды разработки был выбран \texttt{Neovim} \cite{nvim}. %$\vcenter{\hbox{\includegraphics[scale=0.03]{img/nvim.png}}}$.

\subsection{Сведения о модулях программы}

Программа поделена на следующие модули:
\begin{itemize}
    \item \texttt{main.cpp} --- файл, содержащий точку входа в программу, из которой происходит вызов алгоритмов по разработанному интерфейсу;
    \item \texttt{func.cpp} --- файл, содержащий реализации алгоритмов нахождения расстояний Левенштейна и Дамерау --- Левенштейна;
    \item \texttt{test.cpp} --- файл, содержащий модульные тесты и точку входа в программу для их запуска;
    \item \texttt{benchmark.cpp} --- файл, содержащий функции для проведения замеров времени работы реализаций алгоритмов;
    \item \texttt{plot.py} --- файл, содержащий программу для построения графиков по данным о времени выполнения реализаций алгоритмов.
\end{itemize}

\subsection{Реализация алгоритмов}

Листинги исходных кодов программ \ref{lst:cm} -- \ref{lst:dlrec} приведены в приложении.
